%% start of file `template.tex'.
%% Copyright 2006-2013 Xavier Danaux (xdanaux@gmail.com).
%
% This work may be distributed and/or modified under the
% conditions of the LaTeX Project Public License version 1.3c,
% available at http://www.latex-project.org/lppl/.


\documentclass[11pt,a4paper,sans]{moderncv}        % possible options include font size ('10pt', '11pt' and '12pt'), paper size ('a4paper', 'letterpaper', 'a5paper', 'legalpaper', 'executivepaper' and 'landscape') and font family ('sans' and 'roman')

% moderncv themes
\moderncvstyle{banking}                            % style options are 'casual' (default), 'classic', 'oldstyle' and 'banking'
\moderncvcolor{red}                                % color options 'blue' (default), 'orange', 'green', 'red', 'purple', 'grey' and 'black'
%\renewcommand{\familydefault}{\sfdefault}         % to set the default font; use '\sfdefault' for the default sans serif font, '\rmdefault' for the default roman one, or any tex font name
\nopagenumbers{}                                  % uncomment to suppress automatic page numbering for CVs longer than one page

% character encoding
\usepackage[utf8]{inputenc}                       % if you are not using xelatex ou lualatex, replace by the encoding you are using
%\usepackage{CJKutf8}                              % if you need to use CJK to typeset your resume in Chinese, Japanese or Korean
\usepackage{hyphsubst}
\usepackage[english]{babel}
\hyphenation{man-agement}

% adjust the page margins
\usepackage[scale=0.75, top=2cm, bottom=1cm]{geometry}
%\setlength{\hintscolumnwidth}{3cm}                % if you want to change the width of the column with the dates
%\setlength{\makecvtitlenamewidth}{10cm}           % for the 'classic' style, if you want to force the width allocated to your name and avoid line breaks. be careful though, the length is normally calculated to avoid any overlap with your personal info; use this at your own typographical risks...

%\usepackage{xspace}
%\usepackage{multicol}

% personal data
\name{Harald}{Nordgren}
\title{Full-Stack Developer}                               % optional, remove / comment the line if not wanted
%\address{Magistratsvägen 55D}{226 44 Lund}{Sweden}% optional, remove / comment the line if not wanted; the "postcode city" and and "country" arguments can be omitted or provided empty
\address{Lindvägen 23}{192 70 Sollentuna}%{Sweden}
%%\phone[mobile]{+1~(234)~567~890}                   % optional, remove / comment the line if not wanted
%\phone[fixed]{+46 -- 703 -- 16 89 08}                    % optional, remove / comment the line if not wanted
\phone[fixed]{\texttt{+}46--(0)703--16 89 08}
%%\phone[fax]{+3~(456)~789~012}                      % optional, remove / comment the line if not wanted
\email{haraldnordgren@gmail.com}                               % optional, remove / comment the line if not wanted
\homepage{github.com/haraldnordgren}         % optional, remove / comment the line if not wanted
%\extrainfo{github.com/HaraldNordgren}                 % optional, remove / comment the line if not wanted
%%\photo[64pt][0.4pt]{picture}                       % optional, remove / comment the line if not wanted; '64pt' is the height the picture must be resized to, 0.4pt is the thickness of the frame around it (put it to 0pt for no frame) and 'picture' is the name of the picture file
%%\quote{Some quote}                                 % optional, remove / comment the line if not wanted

% to show numerical labels in the bibliography (default is to show no labels); only useful if you make citations in your resume
%\makeatletter
%\renewcommand*{\bibliographyitemlabel}{\@biblabel{\arabic{enumiv}}}
%\makeatother
%\renewcommand*{\bibliographyitemlabel}{[\arabic{enumiv}]}% CONSIDER REPLACING THE ABOVE BY THIS

% bibliography with mutiple entries
%\usepackage{multibib}
%\newcites{book,misc}{{Books},{Others}}
%----------------------------------------------------------------------------------
%            content
%----------------------------------------------------------------------------------
\begin{document}
%\begin{CJK*}{UTF8}{gbsn}                          % to typeset your resume in Chinese using CJK
%-----       resume       ---------------------------------------------------------
\makecvtitle
%\vspace*{-1\baselineskip}
\vspace*{-2\baselineskip}


\section{Experience}
%\subsection{Vocational}
    \cventry{April 2016 -- Present}{Sofware Developer}{Cantemo}{Stockholm, Sweden}{}{I work on website/webserver development with a small team. We do media-asset management, handling video content for ABC, HBO, Disney and Apple. Back-end development in Python (Django/Flask) with microservices (Docker/Kubernetes/Vagrant). Front-end developement in JavaScript (React, Backbone). Elasticsearch API, PostgreSQL, PyCharm.
%\newline{}%
%Detailed achievements:%
%\begin{itemize}%
%\item Achievement 1;
%\item Achievement 2, with sub-achievements:
  %\begin{itemize}%
  %\item Sub-achievement (a);
  %\item Sub-achievement (b), with sub-sub-achievements (don't do this!);
    %\begin{itemize}
    %\item Sub-sub-achievement i;
    %\item Sub-sub-achievement ii;
    %\item Sub-sub-achievement iii;
    %\end{itemize}
  %\item Sub-achievement (c);
  %\end{itemize}
%\item Achievement 3.
%\end{itemize}
}

%\cventry{year--year}{Job title}{Employer}{City}{}{Description line 1\newline{}Description line 2}
%\subsection{Miscellaneous}
%\cventry{year--year}{Job title}{Employer}{City}{}{Description}


\section{Master thesis}
\cvitem{Title}{\emph{Guided Transcoding for Next-Generation Video Coding (HEVC)}}
%\cvitem{Kenneth Andersson}{Supervisors}
%\cvitem{Description}{\\\-\hspace{0.4cm}I am doing research in video compression within the H.265/HEVC standard. I compare methods for effectively storing and transmitting video in varying resolutions for real-time application in streaming services. The project thus far has generated one research article.\\\-\hspace{0.4cm}I wrote my own test scripts for running heavy simulations in a cluster environment, and for collecting large amounts of test data and automatically structuring it in Excel sheets. I use Python, C/C++ and Shell scripts on a day-to-day basis.}

\cvitem{Description}{Research at \textbf{Ericsson} in the H.265/HEVC standard. We investigated methods for reducing complexity and storage requirements for adaptive streaming applications by up to 30\%. I used Python and Platform LSF to run heavy simulations, created automatic Excel exportation of results and wrote C/C\texttt{++} code to manipulate the encoder.}



\section{Education}
\cventry{2009 -- 2016}{Master of Science in Computer Science}{Lund University, Faculty of Engineering (LTH)}{Lund, Sweden}{}{Specializing in Software development. Java, C/C++, Matlab, Shell, Haskell}
\cventry{August 2013 -- May 2014}{Exchange studies}{University of Illinois, College of Engineering}{Urbana--Champaign, Illinois, USA}{}{Mathematics, Control theory, signal processing, databases}
%\cventry{2012 -- 2013}{Film and cultural studies}{Lund University}{Lund, Sweden}{}{}
%\cventry{Aug 2006 -- Jun 2009}{Natural science program}{Heleneholmsgymnasiet}{Malmö, Sweden}{}{High school}


\section{Languages}
%\setlength{\leftskip}{2.5cm}
%\begin{multicols}{2}
\cvitemwithcomment{Swedish}{Native}{}
\cvitemwithcomment{English}{Fluent}{}
%\end{multicols}
%\setlength{\leftskip}{0pt}


%\cvitemwithcomment{Swedish}{Fluent}{My native language}
%\cvitemwithcomment{Language 3}{Skill level}{Comment}

%\section{Programming}
\section{Technologies and frameworks}


%\programmingitem{Python}{}{Django request handling, , Program for finding arbitrages between betting sites (Heroku, Postgre- SQL), ETL between Bandcamp/Spotify and Rateyourmusic (Beautiful Soup, HTML, Splinter/Selenium), Script to probe LTH servers with HTTP requests to find non-indexed exams, Sonic the Hedgehog clone adapted from generic 2D platformer, basic machine learning (scikit-learn)}
%\programmingitem{C/C\texttt{++}}{}{Contributed to Qt project and qBittorrent (Boost, GUI, unit tests, continuous integration), implemented database/newsgroup, wrote malloc/free from scratch (Makefile)}
%\programmingitem{JavaScript/Web}{}{Contributed to Web Scrobbler plugin (Adds Soundcloud data to Last.fm), Chrome extension for automating web games, extension for finding and downloading movies, Django templates, HTML, CSS styling}
%\programmingitem{Shell}{}{Docker scripting, automated builds, git (branching, submodules, rebasing, filtering), ffmpeg (audio and video editing), contributed to oh-my-zsh, config files, ssh, sed, awk, ftp, redirection}
%\programmingitem{Miscellaneous}{}{Java (real-time programming, Swing, Project Euler), Matlab (statistics, video compression), Perl (regex, language analysis), Haskell, Ruby, Batch script, Maple, MIPS Assembly, LaTeX, Excel scripting}



\programmingitem{Python}{}{Django\slash{}Flask (API design, HTTP requests, HTML templates, form data), Elasticsearch, Webscraper for betting arbitrages, migrate music data from Spotify\slash{}Bandcamp to Rateyourmusic (Beautiful Soup, Splinter\slash{}Selenium), Sonic the Hedgehog clone, machine learning}

\programmingitem{JavaScript}{}{React, Backbone, CSS, HTML, Chrome extensions (Movie downloader, automating web games, contributed to Web scrobbler)}

\programmingitem{C/C\texttt{++}}{}{Contributed to qBittorrent (Boost, unit tests) and Qt projects, implemented newsgroup/database, wrote malloc/free from scratch}

\programmingitem{Sysadmin}{}{Shell scripting (curl, ssh, ffmpeg, awk), Clustering (Docker, Kubernetes, Vagrant, Platform LSF), Virtualization (OS X, Ubuntu, Red Hat, CentOS, CoreOS, Alpine, Windows), package management (npm, pip, brew, apt-get), Heroku, databases (PostgreSQL, MySQL)}

\programmingitem{Miscellaneous}{}{Building software (Git, Grunt, Makefile, Jenkins), Java (Concurrency, Swing), Matlab (statistics, image compression), Haskell, Perl (language processing), Batch script, Excel scripting, LaTeX, MIPS Assembly, Ruby, Maple, Project Euler}


%\programmingitem{Advanced}{}{Python, Java, C/C\texttt{++}, Matlab, LaTeX,\\Shell scripting (bash, zsh, tcsh, git,\\vim, grep, ssh, sed, awk, redirection)}
%\programmingitem{Basic skills}{}{JavaScript, HTML, Haskell,\\Perl regex, Batch script, Maple,\\Databases (MySQL, PostgreSQL)}
%\programmingitem{Beginner}{}{Ruby, CSS, MIPS Assembly}
%\cvdoubleitem{category 1}{XXX, YYY, ZZZ}{category 4}{XXX, YYY, ZZZ}

%\section{Interests}
%\cvitem{hobby 1}{Description}
%\cvitem{hobby 2}{Description}
%\cvitem{hobby 3}{Description}

%\section{Extra 1}
%\cvlistitem{Item 1}
%\cvlistitem{Item 2}
%\cvlistitem{Item 3. This item is particularly long and therefore normally spans over several lines. Did you notice the indentation when the line wraps?}

%\section{Extra 2}
%\cvlistdoubleitem{Item 1}{Item 4}
%\cvlistdoubleitem{Item 2}{Item 5\cite{book1}}
%\cvlistdoubleitem{Item 3}{Item 6. Like item 3 in the single column list before, this item is particularly long to wrap over several lines.}

%\section{References}
%Printout from university grades database given on request.
%\begin{cvcolumns}
%  \cvcolumn{Category 1}{\begin{itemize}\item Person 1\item Person 2\item Person 3\end{itemize}}
%  \cvcolumn{Category 2}{Amongst others:\begin{itemize}\item Person 1, and\item Person 2\end{itemize}(more upon request)}
%  \cvcolumn[0.5]{All the rest \& some more}{\textit{That} person, and \textbf{those} also (all available upon request).}
%\end{cvcolumns}

% Publications from a BibTeX file without multibib
%  for numerical labels: \renewcommand{\bibliographyitemlabel}{\@biblabel{\arabic{enumiv}}}% CONSIDER MERGING WITH PREAMBLE PART
%  to redefine the heading string ("Publications"): \renewcommand{\refname}{Articles}

%\nocite{*}
%\bibliographystyle{plain}
%\bibliography{publications}                        % 'publications' is the name of a BibTeX file

% Publications from a BibTeX file using the multibib package
%\section{Publications}
%\nocitebook{book1,book2}
%\bibliographystylebook{plain}
%\bibliographybook{publications}                   % 'publications' is the name of a BibTeX file
%\nocitemisc{misc1,misc2,misc3}
%\bibliographystylemisc{plain}
%\bibliographymisc{publications}                   % 'publications' is the name of a BibTeX file

\clearpage
%-----       letter       -----------------------------------------------------
% recipient data
\recipient{Company Recruitment team}{Company, Inc.\\123 somestreet\\some city}
\date{\today}
\opening{Dear Hiring Manager,}
\closing{Sincerely,}
%\enclosure[Attached]{curriculum vit\ae{}}          % use an optional argument to use a string other than "Enclosure", or redefine \enclname
\makelettertitle

I am a passionate programmer who has been writing code for over ten years. I work with media asset management at Cantemo and I wrote my master's thesis in the area of video compression.

I am a self-starter who likes personal responsibility and I take pride in writing clear and effective code. I know how to work independently -- how to use Google and Stack Overflow to my advantage -- as well as working within a team. I am a social person that thrives in the company of others, I love to travel I would like to pursue and international career at some point.

I enjoy using a multitude of programming languages. Python is my go-to language for any high-level task. I just love how flexible it is and that the syntax reads like plain English. C\texttt{++} and Java are great companions because of the wealth of available libraries. I have experience with low-level C; writing malloc/free from scratch, implementing my own shell and manipulating video bitstreams. I like to use Perl for regex and string manipulation, and I have written a few Chrome extensions in JavaScript. I like to use several languages for the projects and then glue the pieces together.

I have been a Unix user for many years. Ubuntu for many years and the OS X. I love the terminal and spend my spare time writing shell scripts for all kinds of tasks; extracting segments from a media file using ffmpeg, splitting a git repository in two or setting the background image for the terminal window.

As a student I was a high-achieving, with an active interest in mathematics. I tailored my education to be able write as much code as possible and I have knowledge about functional languages, assembly, real-time programming, natural language processing, computer security and algorithm design.

I have been an active member on Rateyourmusic.com for many years, contributing to the their database. As a hoby project I wrote a Python script that scrapes album data from a Bandcamp or Sptify, formats it and automatically navigates RYM to add the album by filling in textboxes and uploading the cover art.

Thanks for your time, I look forward to hearing back from you soon.

\enlargethispage{3\baselineskip}

\makeletterclosing

%\clearpage\end{CJK*}                              % if you are typesetting your resume in Chinese using CJK; the \clearpage is required for fancyhdr to work correctly with CJK, though it kills the page numbering by making \lastpage undefined
\end{document}


%% end of file `template.tex'.
