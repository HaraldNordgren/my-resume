I am a passionate programmer who has been writing code for over ten years. I work with back-end web development and media asset management at Cantemo, and I wrote my master's thesis in the area of video compression.

I am a self-starter who likes personal responsibility and I take pride in writing clear and effective code. I know how to work independently -- how to use Google to my advantage -- as well as being and active member of a team, never afraid to ask for help and always happy to help others. I am a social person that thrives in the company of others, I love to travel would like to pursue and international career at some point.

I enjoy using a multitude of programming languages. Python is my go-to language for any high-level task. I just love how flexible it is and that the syntax reads like plain English. Django and Flask are great tools for web development. I have written a few Chrome extensions in JavaScript and I work with React, Backbone, HTML and CSS on a daily basis in my current position. C\texttt{++} and Java are great companions because of the wealth of available libraries, and I have experience with low-level C. Writing malloc/free from scratch, implementing my own shell and video bitstream manipulations.

I have been a Unix user for many years, first with Ubuntu and now mostly OS X. I love the terminal and often spend my spare time writing shell scripts for all kinds of tasks; extracting segments from a media file using ffmpeg, splitting a git repository in two, or playing around with config files. Every now and then I like to help people on Stack Overflow.

I was always a high-achieving student, with an active interest in mathematics and computer science. I tailored my university education to be able write as much code as possible, and I have knowledge in areas of functional languages, assembly, concurrency, natural language processing, computer security and advanced algorithm theory -- as well as all the basics.

As hobby projects I have contributed to Oh My Zsh, Last.fm Scrobbler, qBittorrent and Qt. I wrote a webscraper to process and compare odds data from bettings sites in order to find sure bets, it was succesful although the potential winning are rather small. As a member of the site Rateyourmusic.com I created a tool for scraping album data from Spotify and Bandcamp URLs to add to RYM by automatically navigating its user interface, filling in textboxes and uploading cover art. Using the Pygame library I have also spent some creating my own 2D platformer games.

Thanks for your time, I look forward to hearing back from you soon!
